\documentclass{cv}

\begin{document}
    \header{Iago Mendes}
    \roundpicture{./img/me.png}
    \firstsection
        {
            \subsection*{\style{Personal Info}}
                \textbf{Full Name} \\ Iago Braz Mendes \\
                \faCalendar \\ December 17th, 2001 \\
                \faHome \\ Montes Claros, MG, Brazil
        }
        {
            \subsection*{\style{Contact}}
                \textbf{Academic Email} \\ {\footnotesize Iago.Braz.Mendes@oberlin.edu} \\
                \textbf{Personal Email} \\ {\footnotesize iagobrazmendes@gmail.com} \\
                \textbf{Phone Number} \\ +55 (38) 9 8404-3111
        }
        {
            \subsection*{\style{Languages}}
                Portuguese (Brazil) \\ English
        }
        {
            \subsection*{\style{Interests}}
                Physics \\ Astronomy \\ Computer Science
        }
    \section*{\style{Standardized Tests}}
        \begin{entrylist}
            \entry
                {2020}
                {First Certificate in English: 184 / 190}
                {\textbf{CEFR Level: C1 (Effective Operational     Proficiency)}}
                {Reading: 184 / 190\\ Use of English: 190 / 190\\ Writing: 182 / 190\\ Listening: 189 / 190\\ Speaking: 174 / 190}
            \entry
                {2019}
                {TOEFL: 103 / 120} 
                {}
                {Reading: 29 / 30\\ Listening: 28 / 30    \\ Speaking: 21 / 30\\ Writing: 25 / 30}
            \entry
                {2019}
                {SAT (Super score): 1390 / 1600} 
                {}
                {Evidence-Based Reading and Writing Score: 600 / 800\\ Math Score: 790 / 800}
            \entry
                {2019}
                {SAT Subjects}
                {}
                {Math Level II: 800 / 800\\ Physics: 770 / 800}
        \end{entrylist}
    \section*{\style{Honors and Awards}}
        \begin{entrylist}
            \entry
                {2019}
                {Bronze Honour}
                {}
                {International Youth Math Competition (IYMC)}
            \entry
                {2019}
                {Bronze Medal}
                {}
                {Brazilian Olympiads of Chemistry (OBQ)}
            \entry
                {2019}
                {Gold Medal}
                {}
                {Brazilian Olympiad of Science (ONC)}
            \entry
                {2019}
                {Gold Medal}
                {}
                {Brazilian Olympiad of Astronomy and Astronautics (OBA)}
            \entry
                {2018}
                {Honorable Medal}
                {}
                {Brazilian Olympiad of Physics (OBF)}
            \entry
                {2018}
                {Honorable Medal}
                {}
                {Brazilian Olympiad of Mathematics from Public and Private Schools (OBMEP)}
            \entry
                {2018}
                {Gold Medal}
                {}
                {Minas Gerais's Olympiad of Chemistry (OMQ)}
            \entry
                {2018}
                {Gold Medal}
                {}
                {Brazilian Olympiad of Astronomy and Astronautics (OBA)}
        \end{entrylist}
    \section*{\style{Volunteering}}
        \begin{entrylist}
            \entry
                {2019 -- Today}
                {Astronomical Olympic League (LOA) - \textit{Content Writer}}
                {Online Platform}
                {\textit{Function:} to provide materials focused on Astronomy Olympiads and to give tips about specific activities.}
            \entry 
                {2020 -- Today}
                {Brazilians in Action for an International Education (BRAIE) - \textit{Mentor}}
                {}
                {\textit{Function:} to help students in the Application process to the United States of America.}
            \entry
                {2020 -- Today}
                {International Astronomy and Astrophysics Competition (IAAC) - \textit{Official Ambassador}}
                {}
                {\textit{Function:} to share about this Scientific Olympiad, inviting students to participate in it.}
            \entry
                {2019 -- 2019}
                {Olympic Scientific Initiation Program (POIC) - {\textit{Founder and President}}}
                {}
                {\textit{Function:} to encourage students to participate in Scientific Olympiads and to help them studying.}
        \end{entrylist}
    \section*{\style{Groups}}
        \begin{entrylist}
            \entry
                {2017 -- Today}
                {Odyssey Group - {\textit{Co-Founder}}}
                {}
                {\textit{Objective:} to spread Science in our community and to help students from public schools.}
            \entry
                {2019 -- Today}
                {Montes Claros's Astronomical Studies Center (CEAMONTES) - {\textit{Member}}}
                {}
                {\textit{Objective:} to spread Astronomy and to make public observations with telescopes.}
        \end{entrylist}
    
    \section*{\style{Research Projects}}
        \begin{entrylist}
            \entry
                {2019 -- Today}
                {Stratospheric Balloon Launch}
                {}
                {The launch of a balloon with sensors and cameras in order to register images and conditions of the stratosphere. It is a project of the Odyssey Group, which has made an association with CEAMONTES. Currently, we are raising money for the project.}
            \entry
                {2019 -- Today}
                {Air Viscosity Study}
                {}
                {A scientific initiation with the objective of determining the air viscosity by filming a cushioned spring-mass oscillation. Currently, we are finishing the article, which will be published soon.}
        \end{entrylist}
    
    \section*{\style{STEM Experience}}
        \begin{entrylist}
            \entry
                {2018 -- 2019}
                {Advanced Astronomy Training - {\textit{Student, 11th position nationally}}}
                {}
                {A training of 5 months (with 3 in-person camps) that selected the top ten students for the International Olympiad of Astronomy and Astrophysics (IOAA) and the Latin-America Olympiad of Astronomy and Astronautics (OLAA).}
            \entry
                {2018}
                {Space Journey Program - {\textit{Student}}}
                {}
                {An event that invited about 90 students from all over Brazil for a week of immersion in Space Subjects, in which we had lectures and visitations.}
            \entry
                {2019}
                {Experimental Classes of Physics - {\textit{Student}}}
                {}
                {A week during winter break in which I had contact with experiences of Physics at a high education level. It was organized by a professor from the local Federal Institute.}
            \entry
                {2019}
                {Aeromodelling Summer Course - {\textit{Student}}}
                {}
                {On December 14th, 2019, I participated in a Summer Course organized by the local aeromodelling group (14 Bits). I spent 8 hours having lectures about their work.}
        \end{entrylist}
    
    \section*{\style{Social Activities}}
        \begin{entrylist}
            \entry
                {2019}
                {Astronomical Meetings - {\textit{Founder and lecturer}}}
                {}
                {Events in which the Odyssey group presents Astronomy for students through lectures and activities, like using a telescope and building rockets from bottles.}
            \entry
                {2019}
                {Youth Conference for Education - {\textit{Organizer}}}
                {}
                {An event in my hometown (Taiobeiras, MG) that showed young people the importance of studying; it counted with lectures about opportunities and benefits of education. It was organized by three partners and me.}
            \entry
                {2019}
                {Lunar Eclipse Event - {\textit{Organizer}}}
                {}
                {In July of 2019, the local astronomy group (CEAMONTES) organized an event to show a lunar eclipse for the public. I was in charge of the telescopes.}
            \entry
                {2019}
                {National Scientific Exposure - {\textit{Lecturer}}}
                {}
                {All public schools in Brazil organized scientific events on November 23rd; I was invited to lecture in a local school about our Solar System.}
        \end{entrylist}
\end{document}