\documentclass{cv}

\begin{document}
    \header{Iago Mendes}
    \roundpicture{./img/me.png}
    \firstsection
        {
            \subsection*{\style{Personal Info}}
                \faTag \\
                    Iago Braz Mendes \\
                \faCalendar \\
                    December 17th, 2001 \\
                \faHome \\
                    Montes Claros, MG, Brazil
        }
        {
            \subsection*{\style{Contact}}
                \faEnvelopeO \\
                    {\footnotesize \href{mailto:Iago.Braz.Mendes@oberlin.edu}{Iago.Braz.Mendes@oberlin.edu}} \\
                    {\footnotesize \href{mailto:iagobrazmendes@gmail.com}{iagobrazmendes@gmail.com}} \\
                \faPhone \\
                    +55 (38) 9 8404-3111 \\
                \faLinkedin $\;$ \href{https://www.linkedin.com/in/iago-mendes-21a2361a2/}{Iago Mendes} \\
                \faGithub $\;$ \href{https://github.com/iago-mendes}{iago-mendes}
        }
        {
            \subsection*{\style{Language proficiency}}
                \faCommentsO \\
                    Portuguese, English \\
                \faCode \\
                    JavaScript, HTML, CSS
        }
        {
            \subsection*{\style{Interests}}
                \faBook \\
                    Cosmology, Astrophysics
                \faDesktop \\
                    Web development, Machine learning
        }
    \section*{\style{Education}}
        \begin{entrylist}
            \entry
                {2020 -- 2024}
                {Oberlin College}
                {Bachelor of Arts -- BA}
                {Major in Physics\\ Major in Computer Science\\ Concentration in Astrophysics}
        \end{entrylist}
    \section*{\style{Standardized Exams}}
        \begin{entrylist}
            \entry
                {2020}
                {First Certificate in English: 184 / 190}
                {CEFR Level: C1 (Effective Operational     Proficiency)}
                {Reading: 184 / 190 (Grade A)\\ Use of English: 190 / 190 (Grade A)\\ Writing: 182 / 190 (Grade A)\\ Listening: 189 / 190 (Grade A)\\ Speaking: 174 / 190 (Grade B)}
            \entry
                {2019}
                {TOEFL: 103 / 120} 
                {}
                {Reading: 29 / 30\\ Listening: 28 / 30    \\ Speaking: 21 / 30\\ Writing: 25 / 30}
            \entry
                {2019}
                {SAT (Super score): 1390 / 1600} 
                {}
                {Evidence-Based Reading and Writing Score: 600 / 800\\ Math Score: 790 / 800}
            \entry
                {2019}
                {Brazilian National High School Exam (ENEM)}
                {}
                {Essay: 980 / 1000\\ Mathematics and its Technologies: 818.7 / 1000\\ Natural Sciences and its Technologies: 721.5 / 1000\\ Human Sciences and its Technologies: 652.0 / 1000\\ Languages, Code and their Technologies: 584.8 / 1000}
            \entry
                {2019}
                {SAT Subjects}
                {}
                {Math Level II: 800 / 800\\ Physics: 770 / 800}
        \end{entrylist}
    \section*{\style{Honors and Awards}}
        \begin{entrylist}
            \entry
                {2020}
                {Ambassador Award for Excellent Encouragement}
                {}
                {Award presented to me for inspiring students about astronomy and encouraging them to participate in the International Astronomy and Astrophysics Competition (IAAC). This award is given to just one official ambassador per country, and I won it in Brazil.}
            \entry
                {2020}
                {Silver Honour}
                {}
                {International Astronomy and Astrophysics Competition (IAAC)}
            \entry
                {2019}
                {Bronze Honour}
                {}
                {International Youth Math Competition (IYMC)}
            \entry
                {2019}
                {Bronze Medal}
                {}
                {Brazilian Olympiads of Chemistry (OBQ)}
            \entry
                {2019}
                {Gold Medal}
                {}
                {Brazilian Olympiad of Science (ONC)}
            \entry
                {2019}
                {Gold Medal}
                {}
                {Brazilian Olympiad of Astronomy and Astronautics (OBA)}
            \entry
                {2018}
                {Honorable Medal}
                {}
                {Brazilian Olympiad of Physics (OBF)}
            \entry
                {2018}
                {Honorable Medal}
                {}
                {Brazilian Olympiad of Mathematics from Public and Private Schools (OBMEP)}
            \entry
                {2018}
                {Gold Medal}
                {}
                {Minas Gerais's Olympiad of Chemistry (OMQ)}
            \entry
                {2018}
                {Gold Medal}
                {}
                {Brazilian Olympiad of Astronomy and Astronautics (OBA)}
        \end{entrylist}
    \section*{\style{Volunteering Experience}}
        \begin{entrylist}
            \entry 
                {2020 -- Today}
                {Mentor}
                {BRAIE -- Brazilians in Action for an International Education}
                {I help students from Brazil to apply to American universities.}
            \entry
                {2020 -- Today}
                {Official Ambassador}
                {IAAC -- International Astronomy and Astrophysics Competition}
                {I disclose the International Astronomy and Astrophysics Competition to students in Brazil and help them in the process.}
            \entry
                {2019 -- Today}
                {Content Director and Writer}
                {LOA -- Astronomical Olympic League}
                {I create free materials for students to study Astronomy, such as classes, exercise lists, solutions for competitions, articles. Besides, as the Content Director, I delegate functions to the Content Team. We have a website, which can be seen \href{https://ligaolimpicadeastronomia.com.br/}{here}.}
            \entry
                {2019 -- Today}
                {Event Organizer, Content Creator, and Lecturer}
                {CEAMONTES -- Montes Claros's Astronomical Studies Center}
                {I have been a part of the organization team and given lectures in some events, such as Eclipse Observation and Astronomy Classes. Besides, since April 2020, I am one of the Content Creators of the social media page.}
            \entry
                {2019}
                {Founder and President}
                {POIC -- Olympic Scientific Initiation Program}
                {I founded this program in my high school to encourage students to participate in Scientific Olympiads, helping to study for them.}
        \end{entrylist}
    \section*{\style{Research Experience}}
        \begin{entrylist}
            \entry
                {2019 -- Today}
                {Air Viscosity Determination}
                {}
                {A scientific initiation with the goal of determining the air viscosity by filming a damped spring-mass oscillation. Currently, we are finishing the article, which will be sent for publication soon.}
        \end{entrylist}
    
    \section*{\style{Science Experience}}
        \begin{entrylist}
            \entry
                {2019}
                {Aeromodelling Summer Course}
                {Student}
                {On December 14th, 2019, I participated in a Summer Course organized by the local aeromodelling group (14 Bits). I spent 8 hours having lectures about their work.}
            \entry
                {2019}
                {Experimental Classes of Physics}
                {Student}
                {A week during winter break in which I had contact with experiences of Physics at a high education level. It was organized by a professor from the local Federal Institute.}
            \entry
                {2018 -- 2019}
                {Advanced Astronomy Training}
                {Student, 11th position nationally}
                {A training of 5 months (with 3 in-person camps) that selected the top ten students for the International Olympiad of Astronomy and Astrophysics (IOAA) and the Latin-America Olympiad of Astronomy and Astronautics (OLAA).}
            \entry
                {2018}
                {Space Journey Program}
                {Student}
                {An event that invited about 90 students from all over Brazil for a week of immersion in Space Subjects, in which we had lectures and visitations.}
        \end{entrylist}
    
    \section*{\style{Social Activities}}
        \begin{entrylist}
            \entry
                {2019}
                {Astronomical Meetings}
                {Founder and lecturer}
                {Events in which a friend and I presented Astronomy for students through lectures and activities, such as using a telescope and building rockets from bottles. During 2019, we organized 6 Astronomical Meetings in mostly public schools in Montes Claros.}
            \entry
                {2019}
                {Youth Conference for Education}
                {Organizer}
                {An event in my hometown (Taiobeiras, MG) that showed young people the importance of studying; it counted with lectures about opportunities and benefits of education. It was organized by three partners and me.}
            \entry
                {2019}
                {Lunar Eclipse Event}
                {Organizer}
                {In July 2019, the local astronomy group (CEAMONTES) organized an event to show a lunar eclipse for the public. I was in charge of the telescopes.}
            \entry
                {2019}
                {National Scientific Exposure}
                {Lecturer}
                {All public schools in Brazil organized scientific events on November 23rd; I was invited to lecture in a local school about our Solar System.}
        \end{entrylist}
\end{document}